%
% File naaclhlt2016.tex
%

\documentclass[11pt,letterpaper]{article}
\usepackage{naaclhlt2016}
\usepackage{times}
\usepackage{latexsym}

%\naaclfinalcopy % Uncomment this line for the final submission
\def\naaclpaperid{***} %  Enter the naacl Paper ID here

% To expand the titlebox for more authors, uncomment
% below and set accordingly.
% \addtolength\titlebox{.5in}    


%todo: possibly a title that reflects the work better? 
\title{Detecting threats of violence in online discussions}

% Author information can be set in various styles:
% For several authors from the same institution:
% \author{Author 1 \and ... \and Author n \\
%         Address line \\ ... \\ Address line}
% if the names do not fit well on one line use
%         Author 1 \\ {\bf Author 2} \\ ... \\ {\bf Author n} \\
% For authors from different institutions:
% \author{Author 1 \\ Address line \\  ... \\ Address line
%         \And  ... \And
%         Author n \\ Address line \\ ... \\ Address line}
% To start a seperate ``row'' of authors use \AND, as in
% \author{Author 1 \\ Address line \\  ... \\ Address line
%         \AND
%         Author 2 \\ Address line \\ ... \\ Address line \And
%         Author 3 \\ Address line \\ ... \\ Address line}
% If the title and author information does not fit in the area allocated,
% place \setlength\titlebox{<new height>} right after
% at the top, where <new height> can be something larger than 2.25in
\author{Author 1\\
	    XYZ Company\\
	    111 Anywhere Street\\
	    Mytown, NY 10000, USA\\
	    {\tt author1@xyz.org}
	  \And
	Author 2\\
  	ABC University\\
  	900 Main Street\\
  	Ourcity, PQ, Canada A1A 1T2\\
  {\tt author2@abc.ca}}

\date{}

\begin{document}

\maketitle

\begin{abstract}
Here comes an abstract for this article
\end{abstract}

\section{Introduction}
Threats of violence is an increasingly common occurrence in online
discussions. It disproportionally affects women and minorities, often
to the point of effectively eliminating them from taking part in
discussions online. Moderators of social networks operate on such a
big scale that manually reading all posts is an insurmountable
task. Methods for automatically detecting threats could therefore
potentially be very helpful, both to moderators of social networks,
and to the members of those networks.

In this article, we \ldots

\section{Previous work}

\section{Data set}

\section{Experiments}

\subsection{Experimental setup}

\subsection{Results}

%\subsection{Error analysis}

%\subsection{Held-out results}

\section{Discussion}

\bibliography{wassa2016}
\bibliographystyle{naaclhlt2016}


\end{document}
